%%%%%%%%%%%%%%%%%%%%%%%%%%%%%%%%%%%%%%%%%
% fphw Assignment
% LaTeX Template
% Version 1.0 (27/04/2019)
%
% This template originates from:
% https://www.LaTeXTemplates.com
%
% Authors:
% Class by Felipe Portales-Oliva (f.portales.oliva@gmail.com) with template 
% content and modifications by Vel (vel@LaTeXTemplates.com)
%
% Template (this file) License:
% CC BY-NC-SA 3.0 (http://creativecommons.org/licenses/by-nc-sa/3.0/)
%
%%%%%%%%%%%%%%%%%%%%%%%%%%%%%%%%%%%%%%%%%

%----------------------------------------------------------------------------------------
%	PACKAGES AND OTHER DOCUMENT CONFIGURATIONS
%----------------------------------------------------------------------------------------

\documentclass[
	12pt, % Default font size, values between 10pt-12pt are allowed
	%letterpaper, % Uncomment for US letter paper size
	%spanish, % Uncomment for Spanish
]{fphw}

% Template-specific packages
\usepackage[utf8]{inputenc} % Required for inputting international characters
\usepackage[T1]{fontenc} % Output font encoding for international characters
\usepackage{mathpazo} % Use the Palatino font

\usepackage{graphicx} % Required for including images

\usepackage{booktabs} % Required for better horizontal rules in tables

\usepackage{listings} % Required for insertion of code

\usepackage{enumerate} % To modify the enumerate environment

\usepackage{xcolor}

\definecolor{mGreen}{rgb}{0,0.6,0}
\definecolor{mGray}{rgb}{0.5,0.5,0.5}
\definecolor{mPurple}{rgb}{0.58,0,0.82}
\definecolor{backgroundColour}{rgb}{0.95,0.95,0.92}

\lstdefinestyle{CStyle}{
    backgroundcolor=\color{backgroundColour},   
    commentstyle=\color{mGreen},
    keywordstyle=\color{magenta},
    numberstyle=\tiny\color{mGray},
    stringstyle=\color{mPurple},
    basicstyle=\footnotesize,
    breakatwhitespace=false,         
    breaklines=true,                 
    captionpos=b,                    
    keepspaces=true,                 
    numbers=left,                    
    numbersep=5pt,                  
    showspaces=false,                
    showstringspaces=false,
    showtabs=false,                  
    tabsize=2,
    language=C
}
%----------------------------------------------------------------------------------------
%	ASSIGNMENT INFORMATION
%----------------------------------------------------------------------------------------

\title{OS 2022 Problem Sheet \#1} % Assignment title

\author{Joshua Law} % Student name

\date{September 15th, 2022} % Due date

\institute{Jacobs University Bremen \\ Bachelor Of Computer Science} % Institute or school name

\class{CO-562 Operating Systems} % Course or class name

\professor{Dr. Jurgen Schonwalder} % Professor or teacher in charge of the assignment

%----------------------------------------------------------------------------------------

\begin{document}

\maketitle % Output the assignment title, created automatically using the information in the custom commands above

%----------------------------------------------------------------------------------------
%	ASSIGNMENT CONTENT
%----------------------------------------------------------------------------------------

\section*{Problem 1.1: \emph{freshie crash}}

\begin{problem}
	A freshmen is learning the C programming language. He wrote the following program but it keeps
	crashing or producing unexpected outputs. Explain why the program crashes or produces unexpected outputs.\end{problem}
	\begin{lstlisting}[style=CStyle]
#include <string.h>
#include <stdio.h>
#include <stdlib.h>

char* strdup(const char *s)
{
	size_t len = strlen(s);
	char d[len+1];
	return strncpy(d, s, len+1);
}

int main(int argc, char *argv[])
{
	int i;

	for (i = 1; i < argc; i++) {
		(void) puts(strdup(argv[i]));
	}

	return EXIT_SUCCESS;
}\end{lstlisting}
%------------------------------------------------

\subsection*{Answer}

On line 5 the function \texttt{strdup} is declared to return a char pointer, and the output of the function is a output of a pointer that points to a variable \texttt{char d[len+1]} on line 8, which would mean that the function returns a pointer that points to something that does not exist in the main function, since function variables are only exculsive to within the function itself.

%----------------------------------------------------------------------------------------

\section*{Problem 1.2: \emph{memory segments}}

\begin{problem}
	Look at the following program and write down what is stored in the text segment, the data segment,
	the heap segment, and the stack segment.\end{problem}
	\medskip
	\begin{lstlisting}[style=CStyle]
#include <stdlib.h>
#include <string.h>
#include <stdio.h>

char *strdup(const char *s)
{
	char *p = NULL;
	size_t len;

	if (s) {
		len = strlen(s);
		p = malloc(len+1);
		if (p) {
			strcpy(p, s);
		}
	}
	return p;
}

int main()
{
	static char m[] = "Hello World!";
	char *p = strdup(m);
		if (!p) {
			perror("strdup");
			return EXIT_FAILURE;
		}
		if (puts(p) == EOF) {
			perror("puts");
			return EXIT_FAILURE;
		}
		if (fflush(stdout) == EOF) {
			perror("fflush");
			return EXIT_FAILURE;
		}
	return EXIT_SUCCESS;
}\end{lstlisting}
%------------------------------------------------
\subsection*{Answer}
The text segment contains machine instructions, the data segment contains the static variable \texttt{m[]}, the heap segment contains the char variable \texttt{*p}, the stack segment contains the function parameters such as \texttt{const char *s} as a parameter of \texttt{*strdup} on line 5, the parameter in \texttt(perror("strerr")) also contributes to the stack, return addresses such as \texttt{EXIT\_FAILURE} on line 26 and line 30 are also within stack. 

\section*{Problem 1.3: \emph{execute a command in a modified environment or print the environment}}

\begin{problem}
	On Unix systems, processes have access to environment variables that can influence the behavior
	of programs. The global variable \texttt{environ}, declared as\newline

	\texttt{	extern char **environ;}	\newline

	points to an array of pointers to strings. The last pointer has the value \texttt{NULL}. By convention, the
	strings have the form “name=value” and the names are often written using uppercase characters.
	Examples of environment variables are \texttt{USER} (the name of the current user), \texttt{HOME} (the current
	user’s home directory), or \texttt{PATH} (the colon-separated list of directories where the system searches
	for executables).\newline
	Write a program env that implements some of the functionality of the standard \texttt{env} program. The
	syntax of the command line arguments is the following:\newline

	\texttt{	env [OPTION]... [NAME=VALUE]... [COMMAND [ARG]...]}\newline
	\medskip
	
	\begin{enumerate}[(\itshape a\normalfont)] % Sub-questions styled as italic letters
		\item If called without any arguments, \texttt{env} prints the current environment to the standard output.
		\item If called with a sequence of “name=value” pairs and no further arguments, the program adds
		the “name=value” pairs to the environment and the prints the environment to the standard
		output.
		\item If called with a command and optional arguments, \texttt{env} executes the command with the given
		arguments.
		\item If called with a sequence of “name=value” pairs followed by a command and optional arguments, the program adds the “name=value” pairs to the environment and executes the command with the given arguments in the modified environment.
		\item If called with the option \texttt{-v}, the program writes a trace of what it is doing to the standard error.
		\item If called with the option \texttt{-u} name, the program removes the variable name from the environment.
	\end{enumerate}\end{problem}

%------------------------------------------------

\subsection*{Answer} 

env.c


%----------------------------------------------------------------------------------------


\end{document}
